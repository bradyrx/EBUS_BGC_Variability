\documentclass[]{article}

%opening
\title{Controls on Sea-Air CO$_{2}$ Flux in EBUS}
\author{Riley X. Brady}
\date{\today}
\setlength\parindent{0pt}
\usepackage{natbib}
\bibliographystyle{plainnat}

\begin{document}

\maketitle
\begin{abstract}
\noindent Working to understand what controls historical variability in Sea-Air CO$_{2}$ Flux in Eastern Boundary Upwelling Systems. I use FG\_CO2 output from the CESM Large Ensemble and correlate it to various climate indices derived from model output.
\end{abstract}

\section{California Current}

\subsection{Study Site}
For simplicity, I am using the latitudinal bounds set up by \citet{Chavez:2009}. This equates to 34N - 44N for the CCS. In terms of longitude, I want to approach it similarly to \citet{Turi:2014}, generating a 100km band (0-100km), a near-offshore band (100km-400km), and a far-offshore band (400km-800km). I can use the lon\_bounds that comes with the model output (since longitude varies with latitude). This can help to standardize the sections we're looking at and differentiate between coastal and curl-driven upwelling areas.

\bibliography{../../EBUS_BGC_Bibliography.bib}
\end{document}
