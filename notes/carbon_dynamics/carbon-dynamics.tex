\documentclass[12pt]{article}
\usepackage[margin=1in]{geometry}     
\usepackage{graphicx}
\usepackage{epstopdf}
\setlength\parindent{0pt}
\usepackage{natbib}
\usepackage{booktabs}
\bibliographystyle{plainnat}
\usepackage{soul}
\usepackage{subcaption}
\usepackage{booktabs}
\usepackage{threeparttable}
\usepackage{rotating}
\usepackage{pdflscape}
\usepackage{color}
\usepackage{xcolor}

\usepackage[format=plain,
labelfont=it,
textfont=it]{caption}

% Commands
\newcommand{\pco}{p$CO_{2}$}
\newcommand{\bicarb}{HCO$_{3}^{-}$}
\newcommand{\carb}{CO$_{3}^{2-}$}
\newcommand{\carbonic}{H$_{2}$CO$_{3}$}

%opening
\title{Ocean Carbon Dynamics}

\begin{document}

\maketitle
Notes from \it Ocean Biogeochemical Dynamics\rm. Adapted to fit the CESM model.

\section{Air-Sea Interface}
The net sea-air flux is directly proportional to the partial pressure difference across the air-sea interface or its corresponding concentration anomaly.
$$
	\Delta p^{CO_{2}} = p^{CO_{2}}_{w} - p^{CO_{2}}_{a}
$$
$$
	\Delta[CO_{2}] = [CO_{2}]_{w} - [CO_{2}]_{eq}
$$

Henry's law connects these two values via a solubility term that is dependent on temperature and salinity. NOTE that this is the 'inverse' of the conventional chemistry Henry's Law. Technically, we are using k$^{\prime}$.
$$
[CO_{2}]_{eq} = S_{CO_{2}}\cdot p_{CO_{2}}^{atm}
$$

In either case, a supersaturated area of the ocean---either $\Delta p^{CO_{2}}$ or $\Delta[CO_{2}]$ $>$ 0---will lead to outgassing of that carbon from the ocean into the atmosphere. An undersaturated area will lead to uptake. \\

The flux of CO$_{2}$, $F_{CO_{2}}$, is related to the concentration anomaly and a `piston velocity':
$$
	F_{CO_{2}} = k\cdot S_{CO_{2}}\cdot(p_{CO_{2}}^{oc} - p_{CO_{2}}^{atm})
$$

{\color{red} {\bf Note:}~There is an apparent contradiction in the flux equation. One might notice from section 2.2 that the \pco$^{oc}$ equation contains K$_{0}$---or S$_{CO_{2}}$---in the denominator. Why do we use the solubility equation again? It's actually a manifestation of Henry's Law. Since [CO$_{2}$] = S$_{CO_{2}}$pCO$_{2}$, we can change the flux equation to the following:
$$
	F_{CO_{2}} = k\cdot S_{CO_{2}}\cdot (\frac{[CO_{2}]_{oc}}{S_{CO_{2}}} - \frac{[CO_{2}]_{eq}}{S_{CO_{2}}})
$$	

This backs out the concentration anomaly, which flux is apparently governed by.} \\

In CESM, the solubility term (coined `ff') is a complex polynomial based on salinity, temperature, and a bunch of coefficients. The calculation can be found in Nikki's MATLAB code, and its general formation (with different coefficients) can be found in Table 3.2.2 of the Ocean BGC text. \\

The piston velocity in CESM follows the Wanninkhof [1992] parameterization, accounting for the influence of ice:
$$
	0.31\sqrt{\frac{660}{Sc}}U^{2}_{10}(1 - ifrac)
$$

Recall that the Schmidt number is a function of temperature and salinity. \\

\section{Carbon Cycle}
Let's start with a huge term in the CO$_{2}$ flux equation: pCO$_{2}$ in the surface ocean. Since CO$_{2}$ reacts with seawater and is stored away as various species, the ocean holds 60 times more carbon than the atmosphere. It thus plays a large role in modulating atmospheric CO$_{2}$. Since atmospheric CO$_{2}$ is relatively uniform globally at the surface, the variability in $\Delta p^{CO_{2}}$ is largely driven by variations in surface ocean $p^{CO_{2}}$. Oceanic \pco is controlled by SST, salinity, DIC (dissolved inorganic carbon), and alkalinity. \\

\subsection{Inorganic Carbon Chemistry}
CO$_{2}$ is a unique gas, as it reacts with seawater itself, acting as an acid proton (H$^{+}$) donor. When CO$_{2}$ enters the water, it first reacts with H$_{2}$O to become an aqueous solution (CO$_{2(aq)}$). The aqueous gas then reacts with H$_{2}$O again to form carbonic acid. It is difficult to analytically differentiate CO$_{2(aq)}$ from carbonic acid, so a hypothetical species, H$_{2}$CO$_{3}$$^{\ast}$ has been created.	\\

The carbonic acid then dissociated in two steps to form bicarbonate (\bicarb) and carbonate (\carb). These reactions happen very fast, so for all practical purposes, we can assume that the carbonate system is in thermodynamic equilibrium (unlike the relationship between \pco in the ocean and atmosphere). \\

The equilibrium relationships between all the species illuminate the complexity of the carbonate system (note that $K_{0}$ is the same as solubility):
$$
K_{0} = \frac{[H_{2}CO_{3}^{\ast}]}{pCO_{2}^{atm}}
$$
$$
K_{1} = \frac{[H^{+}][HCO_{3}^{-}]}{[H_{2}CO_{3}^{\ast}]}
$$
$$
K_{2} = \frac{[H^{+}][CO_{3}^{2-}]}{[HCO_{3}^{-}]}
$$

The CO$_{2}$ system at present thus comprises of 5 unknowns (\pco, [H$^{+}$], [\bicarb], [\carb], [\carbonic]) and the three above equations. One thus needs to know two unknowns to solve the system of equations. In the real world, \pco and pH (thus $H^{+}$) are frequently measured, but depend on temperature, salinity, and pressure. Thus, models tend to rely upon conservative measurements, such as DIC and Alk. \\

DIC is the total dissolved inorganic carbon:
$$
DIC = [H_{2}CO_{3}^{\ast}] + [HCO_{3}^{-}] + [CO_{3}^{2-}]
$$

Alk is the total bases over acids:
$$
Alk = [HCO_{3}^{-}] + 2[CO_{3}^{2-}] + [OH^{-}] + [B(OH)_{4}^{-}] + \textrm{minor bases} - [H^{+}]
$$

Most of the alkalinity power and most carbon speciation is in the form of carbonate and bicarbonate. Thus, Alk and DIC can be reframed by the following:
$$
DIC \approx [HCO_{3}^{-}] + [CO_{3}^{2-}]
$$
$$
Alk \approx [HCO_{3}^{-}] + 2[CO_{3}^{2-}]
$$

Now framing carbonate and bicarbonate in terms of the conservative properties:
$$
[CO_{3}^{2-}] = Alk - DIC
$$
$$
[HCO_{3}^{-}] = 2\cdot DIC - Alk
$$

These approximations help us to understand the peculiarities of CO$_{2}$ in seawater and help to develop rules of thumb for the carbonate system (that ultimately has 10 unknowns and 8 equations). Lastly, sDIC and sAlk are sometimes used with reference to normalizing these properties for salinity changes. Freshwater flux (precipitation/evaporation) at the ocean surface can dilute DIC and Alk, thus washing out the signal of biology, chemistry, and circulation that we are interested in.

\subsection{Surface Ocean pCO$_{2}$}
In the simplest sense, surface \pco is determined by the concentration of \carbonic and the solubility of CO$_{2}$ (K$_{0}$):
$$
pCO_{2} = \frac{[H_{2}CO_{3}^{\ast}]}{K_{0}}
$$

But in reality, since such a small amount of carbon is in the carbonic acid/carbon dioxide pool, \pco is modulated by the equilibrium constants, carbonate, and bicarbonate. We can rearrange some of the former equations to show:
\begin{equation}
pCO_{2} \approx \frac{K_{2}}{K_{0}K_{1}}\frac{(2\cdot DIC-Alk)^{2}}{Alk - DIC}
\end{equation}

Complicated again! Let's investigate what controls the three parts of the equation: the ratio of equilibrium constants, DIC, and Alk. It's best to think in terms of physical processes and biological processes. DIC can be altered through air-sea gas exchange and biological processes, while Alk can only be altered through biological processes.

\subsubsection{Temperature}
At a constant DIC and Alk, increased temperature causes an increase in \pco. In a simplistic sense, the added thermal energy causes an increase in pressure from the energetic molecules. K$_{0}$ (the solubility term) sits on the bottom of Equation 1. As the temperature increases, K$_{0}$ gets smaller, causing \pco~to increase. Another way to think of it is that atmospheric \pco~stays constant, but the solubility term drops, so [CO$_{2}$]$_{eq}$ drops. Since our [CO$_{2}$]$_{oc}$ stays constant, we have a larger anomaly. \\

The sensitivity of \pco~to temperature is based on a constant and the current partial pressure:
$$
\frac{\partial pCO_{2}}{\partial T} = 0.0423^{o}C^{-1}\cdot pCO_{2}
$$

\subsubsection{Salinity}
An increase in salinity increases \pco. Most of the K$_{0}$ term is from temperature. Thus, salinity makes its mark by increasing the K$_{2}$ to K$_{1}$ ratio. \\

The sensitivity of \pco to salinity is also a simple empirical relationship:
$$
\frac{\partial pCO_{2}}{\partial S} = \frac{pCO_{2}}{S}
$$

Note that salinity changes are largely driven by freshwater fluxes (evaporation and precipitation), so one can consider normalizing DIC and Alk for salinity effects.

\subsubsection{Air-Sea Gas Exchange}
An input of CO$_{2}$ into the surface ocean causes DIC to increase, but causes no relative change to Alk. \bf Gas exchange does not influence the charge balance of the surface\rm. If a CO$_{2}$ molecule speciates to bicarbonate, it produces a proton but also the \bicarb is a one proton base. If it speciates to carbonate, it produces protons, but the \carb is a two proton base. 

\end{document}
